\documentclass[12pt]{article}
\usepackage{minted}
\usepackage[pdftex]{graphicx}
\usemintedstyle{emacs}

%\usepackage[margin=1.25in]{geometry}
\usepackage[onehalfspacing]{setspace}
%opening
\title{Task \#3 part 2\\Implementing the CKY algorithm}
\author{Thuong-Hai Pham}

\begin{document}

\maketitle

\section{Source code overview}
Beside \textbf{main()} function, which reads the grammar, input file and does preliminary checking, the main part of our program can be divided into two parts: \textbf{parse()} (parsing) and \textbf{back\_track()} (backtracking).

\subsection{parse()}
As a tradition to Dynamic Programming algorithm, we need a table to memorise previous results. In this program, we use a 2D table: \mint{python}{f[i][j] = {label: [(k, prod)...]...}}
in which:
\begin{itemize}
	\item \textbf{i, j} denote the sentence segment that our current cell coverages (from word $i$ to word $j-1$)
	\item Each cell maintains list of key-value pair \textbf{label: [(k, prod)...]} (for backtracking purpose)
	\begin{itemize}
		\item \textbf{label} is the current node label (i.e. left-hand side (lhs) of the production rule), for faster search from parent cells
		\item \textbf{k} denotes the separation index, that divided this current segment into two sub-segments $(i, k)$ and $(k, j)$
		\item \textbf{prod} stores the production rule that does the splitting (with labels of the two child nodes, or one child node if $k$ is None). This is required because a triplet $(i, k, j)$ can be produced by different production rules (composed by various child node labels)
	\end{itemize}
\end{itemize}

Having our table, we first fill it with production rules lead to a terminal. To be more specific, we fill all $f[i][i+1]$ cells ($i < l$: length of the input sentence). Then, we follow these steps:
\begin{enumerate}
	\item Generate all triplet $(i, k, j)$, $(i<k<j)$
	\item Search for production rules that matches our condition defined by CKY algorithm
	\item Add new valid separation index (k) and production rule to current cell
	\item Until we reach $f[0][l]$
\end{enumerate}

\subsection{back\_track()}
In backtracking part, we implement \textbf{go()} function that traverses through the result table in Depth-First-Search approach.
\begin{enumerate}
	\item Start with (0, $l$, grammar.start) (the only valid top node)
	\item If the node can not be divided ($k$ is None, production: Non-terminal -\textgreater terminal), then return a single tree: Non-terminal -\textgreater terminal.
	\item else, get two lists of left and right possible sub-trees by traversing to left, right node, consequently, then, return a Cartesian product of those two lists.
\end{enumerate}

\section{Problems and Solutions}
Grammar provided is not minimised, some rules are duplicated. As a consequence, the output has some identical trees.
\begin{minted}[linenos,firstnumber=102]{text}
Nom -> X15 Nom
Nom -> Nom PP	# duplicated
Nom -> Nom NP
Nom -> Adj Nom
Nom -> Nom PP	# duplicated
Nom -> ADJP Nom
\end{minted}

To solve this problem, when querying production rules from grammar, we need to remove duplicated rule by using ``set" as below
\begin{minted}[linenos,firstnumber=27]{python}
for _prod in set(grammar.productions(rhs=lhs_1)):
\end{minted}
instead of
\begin{minted}[linenos,firstnumber=27]{python}
for _prod in grammar.productions(rhs=lhs_1):
\end{minted}


\end{document}