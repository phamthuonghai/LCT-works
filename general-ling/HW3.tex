\documentclass{article}
\usepackage[utf8]{inputenc}
\usepackage[vietnamese]{babel}

\title{NPFL063 - HW 3}
\author{Thuong-Hai Pham}
\date{January 2017}

\usepackage{qtree}

\begin{document}

\selectlanguage{vietnamese}
\maketitle

\section*{Find 5 examples of some imported syntactic structure in your language}

Unlike in English or Mandarin, in Vietnamese, the modifier always follows the modified word (head-initial). For example, a noun can only be modified by succeeding adjectives/nouns:

\begin{itemize}
    \item \textit{Rượu} (wine) \textit{đỏ} (red): red wine
    \item \textit{Màn hình} (screen) \textit{tivi} (TV): TV screen
\end{itemize}

However, in some cases, Vietnamese speakers use the construction of head-final, which was borrowed from Mandarin, especially for words that have Chinese origin (Sino-Vietnamese words):

\begin{itemize}
    \item \textit{Nữ} (female) \textit{phóng viên} (journalist): female journalist (traditional Vietnamese: \textit{``phóng viên nữ"})
    \item \textit{Nhóm} (team) \textit{trưởng} (leader): team leader (traditional Vietnamese: \textit{``trưởng nhóm"})
    \item \textit{Đại} (big) \textit{lộ} (street): boulevard (\textit{``lộ đại" } is  not used)
    \item \textit{Bê tông} (concrete, from the French word ``béton") \textit{hóa} (transformation): the process of replacing a construction (road, bridge, house) with a concrete version. (\textit{``hóa bê tông" } is  a verb phrase)
    \item \textit{Yếu} (important) \textit{điểm} (point): important point. (\textit{``điểm yếu"}  means ``weak point")
\end{itemize}



\end{document}
