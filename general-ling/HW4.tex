\documentclass{article}
\usepackage[utf8]{inputenc}

\title{NPFL063 - HW 4}
\author{Thuong-Hai Pham}
\date{February 2017}

\begin{document}

\maketitle

\section*{Find three advertisements/product labels, where the information really meant by the producer is different from the conventional implicature}

.

\textbf{``A 9-year-old fully functional laptop (Japanese domestic product) for 1 million VND."} (in 2011)

\textbf{Conventional implicature:} This is a quite old laptop, yet still fully functional and it is from Japan, so quite durable. 1 million is a steal as a new one would cost 10 times more, and used laptop costs no less than 3 million.

\textbf{Reality:} The laptop does not come with its adapter. In addition, the model is so out-dated that there is no accessory in the market, only that store has a compatible adapter, which costs the same price of the laptop.

\bigskip

\textbf{``Collect 6/6 different figures from ABC instant noodle cups to win free ABC for a year."}

\textbf{Conventional implicature:} It is common to believe that after opening less than 30 (or other reasonably small numbers of) noodle cups, which is achievable, one can collect all of the 6 figures.

\textbf{Reality:} However, two of the figures are so scarce that the number of winners were relatively small. And the number of customers who tried until realizing the fact were large enough to significantly increase the sales.

\bigskip

\textbf{``XYZ has 30\% more cleaning power."}

\textbf{Conventional implicature:} 30\% more cleaning power than the competitors, or at least the previous version of this detergent.

\textbf{Reality:} The comparison was made against water, without any cleansing agent.

\end{document}
