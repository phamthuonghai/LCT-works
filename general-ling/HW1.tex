\documentclass{article}
\usepackage[utf8]{inputenc}
\usepackage{tipa}

\title{NPFL063 - HW 1}
\author{Thuong-Hai Pham}
\date{November 2017}

\usepackage{graphicx}

\begin{document}

\maketitle

\section{Are \textipa{[\c{c}]} and \textipa{[x]} allophones of the same phoneme?}

One would argue that \textipa{[\c{c}]} and \textipa{[x]} are allophones of the same phoneme because:
\begin{enumerate}
    \item \textipa{[\c{c}]} and \textipa{[x]} are closely related in term of phonetics, both are \textit{voiceless fricative} consonants. While \textipa{[\c{c}]} is \textit{palatal}, \textipa{[x]} is \textit{velar}.
    \item According to the data, \textipa{[\c{c}]} and \textipa{[x]} are mutually exclusive. \textipa{[x]} appears after \textit{back} vowels (\textipa{[A,u,o,o:]}), while \textipa{[\c{c}]} appears after \textit{front} vowels (\textipa{[i,e,E]}) and consonants (\textipa{[d,l]}).
\end{enumerate}

\section{Realizing rule}

As mentioned above, the rule should be assimilation, when the preceding sound is a \textit{back} vowel, use \textipa{[x]} instead of \textipa{[\c{c}]} as it is easier to pronounce (\textit{back} vowel and \textit{velar} consonant). Formally: \textipa{/\c{c}/} $\rightarrow$ \textipa{[x]} / [+back]\_

\end{document}
