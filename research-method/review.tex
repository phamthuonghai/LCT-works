\documentclass[conference,compsoc]{IEEEtran}
% *** MISC UTILITY PACKAGES ***
%
%\usepackage{ifpdf}
% \ifpdf
%   % pdf code
% \else
%   % dvi code
% \fi
% \ifCLASSINFOpdf conditional that works the same way.
\usepackage{graphicx}

% *** CITATION PACKAGES ***
%
\ifCLASSOPTIONcompsoc
% IEEE Computer Society needs nocompress option
% requires cite.sty v4.0 or later (November 2003)
\usepackage[nocompress]{cite}
\else
% normal IEEE
\usepackage{cite}
\fi

% *** GRAPHICS RELATED PACKAGES ***
%
\ifCLASSINFOpdf
% \usepackage[pdftex]{graphicx}
% \graphicspath{{../pdf/}{../jpeg/}}
% \DeclareGraphicsExtensions{.pdf,.jpeg,.png}
\else
% \usepackage[dvips]{graphicx}
% \graphicspath{{../eps/}}
% \DeclareGraphicsExtensions{.eps}
\fi




% *** MATH PACKAGES ***
%
\usepackage{amsmath}
%\interdisplaylinepenalty=2500




% *** SPECIALIZED LIST PACKAGES ***
%
%\usepackage{algorithmic}



% *** ALIGNMENT PACKAGES ***
%
%\usepackage{array}


% IEEEtran contains the IEEEeqnarray family of commands that can be used to
% generate multiline equations as well as matrices, tables, etc., of high
% quality.




% *** SUBFIGURE PACKAGES ***
\ifCLASSOPTIONcompsoc
\usepackage[caption=false,font=footnotesize,labelfont=sf,textfont=sf]{subfig}
\else
\usepackage[caption=false,font=footnotesize]{subfig}
\fi



% *** FLOAT PACKAGES ***
%
%\usepackage{fixltx2e}

%\usepackage{stfloats}
%\fnbelowfloat
% \baselineskip 
% \usepackage{dblfloatfix}




% *** PDF, URL AND HYPERLINK PACKAGES ***
%
%\usepackage{url}




% correct bad hyphenation here
\hyphenation{}


\begin{document}
	%
	% paper title
	% Titles are generally capitalized except for words such as a, an, and, as,
	% at, but, by, for, in, nor, of, on, or, the, to and up, which are usually
	% not capitalized unless they are the first or last word of the title.
	% Linebreaks \\ can be used within to get better formatting as desired.
	% Do not put math or special symbols in the title.
	\title{Paper Review: ``Latent Dirichlet Allocation"}
	
	
	% author names and affiliations
	% use a multiple column layout for up to three different
	% affiliations
	\author{\IEEEauthorblockN{Thuong-Hai Pham}
		\IEEEauthorblockA{Faculty of Information and Communication Technology\\
			University of Malta\\
			Msida MSD 2080, Malta\\
			Email: thuong-hai.pham.16@um.edu.mt}
	}
	
	% conference papers do not typically use \thanks and this command
	% is locked out in conference mode. If really needed, such as for
	% the acknowledgment of grants, issue a \IEEEoverridecommandlockouts
	% after \documentclass
	
	% for over three affiliations, or if they all won't fit within the width
	% of the page (and note that there is less available width in this regard for
	% compsoc conferences compared to traditional conferences), use this
	% alternative format:
	% 
	%\author{\IEEEauthorblockN{Michael Shell\IEEEauthorrefmark{1},
	%Homer Simpson\IEEEauthorrefmark{2},
	%James Kirk\IEEEauthorrefmark{3}, 
	%Montgomery Scott\IEEEauthorrefmark{3} and
	%Eldon Tyrell\IEEEauthorrefmark{4}}
	%\IEEEauthorblockA{\IEEEauthorrefmark{1}School of Electrical and Computer Engineering\\
	%Georgia Institute of Technology,
	%Atlanta, Georgia 30332--0250\\ Email: see http://www.michaelshell.org/contact.html}
	%\IEEEauthorblockA{\IEEEauthorrefmark{2}Twentieth Century Fox, Springfield, USA\\
	%Email: homer@thesimpsons.com}
	%\IEEEauthorblockA{\IEEEauthorrefmark{3}Starfleet Academy, San Francisco, California 96678-2391\\
	%Telephone: (800) 555--1212, Fax: (888) 555--1212}
	%\IEEEauthorblockA{\IEEEauthorrefmark{4}Tyrell Inc., 123 Replicant Street, Los Angeles, California 90210--4321}}
	
	
	
	
	% use for special paper notices
	%\IEEEspecialpapernotice{(Invited Paper)}
	
	
	
	
	% make the title area
	\maketitle
	
	% As a general rule, do not put math, special symbols or citations
	% in the abstract
	\begin{abstract}
		To deal with the explosion of data these days, especially electronic documents and text generated by world-wide-web users demands techniques that automatically organized large collection of text. One family of those techniques is called ``topic model". These techniques discover underlying topic from a given corpus with or without intervention of human, in other word, supervised and unsupervised. This report is aimed to give a review of Blei et al. work in these techniques family, Latent Dirichlet Allocation, its suitability to be published in the Journal of Machine Learning Research 3, 2003, and to propose a technical improvement that should overcome the problem remained.
	\end{abstract}
	
	% no keywords
	
	\section{Suitability of the topic}
	\subsection{Topic appealing to the Journal readers}
	The paper was submitted to be published in the Journal of Machine Learning Research 2003. 
	By eliminating the assumptions in tf-idf, Latent Semantic Index (LSI) and probability Latent Semantic Index (pLSI), Latent Dirichlet Allocation (LDA, Blei et al. \cite{Blei2003}) makes use of De Finite theorem as the base idea and works as an unsupervised method to extract underlying topics.
	
	Although the topic seems to be more relevant to Natural Language Processing (NLP), the proposed method, in conjunction with the previous methods in the family such as LSI, plays an important role in Information Retrieval (IR) when applying to a general collection of data, not just text documents. Hence, the topic is appropriate and useful for the readers of the Journal of Machine Learning Research as well.

	\subsection{Impact in the field}
	In the reviewer opinion, the topic of Blei's paper will provide an efficient tool for researchers in Information Retrieval, in general, and Natural Language Processing to extract information from a large collection of documents. As mentioned above, despite the fact that the paper uses text documents to illustrate the idea, the proposed method is appropriate to apply in a more general context of Information Retrieval. Thus, most of the works which are currently depending on the algorithm family (LSI, pLSI, LSA) can be re-evaluated with the new method. More important, after being approximated, the posterior in LDA can be used in many other applications such as collaborative filtering, document similarity, etc.
	
	
	\section{Content}

	The paper involves in discussing the topic by both examining deeply into the proposed method and covering a wide range of related mathematical concepts and theorems backing up the method.
	
	\subsection{Coverage of the topic}
	The proposed method, which is also the main topic that the authors focus on in this paper, is covered sufficiently in the $3^{rd}$ section (Latent Dirichlet Allocation). To prepare for the readers adequate background to understand that, the authors also mention the mathematical concepts of De Finite theorem in the same section.
	
	\subsection{Technical depth}
	By discussing inference and parameters estimation in detail, the authors have not only proposed a method which is solely a combination or improvement of the previous works but have proposed a new idea of a working mixture models and how to solve the latent variables inference problem. Even though the inference method used in the paper is not the authors work on their own, the authors have applied it wisely in the context of the work. Moreover, the authors have proved the effectiveness of the proposed method on various tasks with carefully chosen measurements. These tasks vary from topic model to text classification, which is proof of the statement the authors mention as its wide application in IR.
	
	All in all, the paper suits a wide range of readers from nonspecialist (e.g. linguists without computational background) to expert in the field (to improve the inference methods and to modify the LDA variables model).
	
	\subsection{Technical novelty}
	The proposed method is not an improvement of any previous algorithm in the family, which kind of works are very common in the field of Machine Learning Research. In contrast, the work is purely a new idea of mixture model based on observation of exchangeability and bag-of-word assumption. By having that, the method works well on both word mixture and more complicated structures such as documents or paragraphs. In addition to the method itself, the author emphasize the novelty of the proposed method by comparing it with the other related probabilistic models in Section 4. The experiment result reported in Section 7 also clearly prove the effectiveness and innovation of the method.
	
	\subsection{Authoritative and originality}
	In general, the paper can be considered to be authoritative and provide a content with high quality and originality. The paper clearly distinguishes the authors' contribution (the model, observation...) and applying colleagues' works (the inference methods).
	
	\section{Presentation}
	\subsection{Overall organization}
	The paper is well organized with 6 separated main parts: introduction of the previous methods, notation and terminology used in the paper, the proposed method, relationship with other probabilistic models, inference and parameter estimation (which is crucial for this kind of Bayesian model), and empirical results. Besides that, the length of the paper which is 30 pages fits the article length constraint of the Journal of Machine Learning Research.
	
	\subsection{Title and abstract}
	The title is short with only the name of the method. However, the abstract provides an ample amount of information to support the title, by first describing the method itself as ``a generative probabilistic model for collections of discrete data" \cite{Blei2003} which clearly states all the characteristics of the method and the domain that it can be applied to. In this part, the author also mentions the important method to infer this type of Bayesian model and briefly introduces how experiment is constructed with the counterpart method. In general, the abstract is adequately written to support the short title, which together provide a clear, accurate indication of the material discussed throughout the paper.
	
	\subsection{Symbols, terms, and concepts' definition}
	Not only adequately define all the concepts and terms used in the paper, the authors separate these into a single section after introduction. In which, they also formally define the relationship between these terms and concepts.
	
	\subsection{English usage}
	The authors have used proper and comprehensive academic English to convey the necessary information.
	
	\subsection{Bibliography}
	In the bibliography, the authors have cited all the related works mentioned throughout the article. These works are ranging from not only previous methods (LSI, td-idf, pLSI) but also mathematical concepts and proofs (De Finite theorem, exchangeability) which references are complete and accurate.
	
	\section{Overall comments \& recommendations}
	Having analyzed in detail the work of Blei et al. as stated in the sections above, it is obviously that the paper uses an excellent literary style and has high quality and originality. Importantly, nonspecialist from other fields can easily access mostly the knowledge conveyed by the paper with high tutorial value. By claiming that value does not reject the impact of the paper for specialists in Information Retrieval and Natural Language Processing due to the fact that LDA provides a mixture model and very efficient tool which is easily customized for improvement and adapt other contexts.
	
	In conclusion, this paper is excellent and should be published in the Journal of Machine Learning Research.
	
	\section{Technical improvement}
	The proposed method has been showed to outperform the other approaches in the family.	However, we argue that the method may fail to model topic(s) of short-text documents, such as Twitter's tweets, Q\&A questions or other social media generated contents, due to the constraint of text length (140 characters in case of Twitter). In addition, these are becoming huge sources of data to be analyzed and applying topic model for micro-blogging site is a very important task to enhance our understanding of the social network. Hence, the two following scenarios should be the future works to solve the problem without modifying the original LDA process:
	\begin{enumerate}
		\item Applying LDA on a single user, each tweet as a document, then group these documents based on their LDA-topics.
		\item Grouping tweets based on their tweeted/retweeted timestamp.
	\end{enumerate}
	The above approaches take Twitter's tweets as the example to do the experiment. After processing the corpus by one of these above mentioned methods, we treat the grouped tweets as a single document to apply LDA.

	% An example of a floating figure using the graphicx package.
	% Note that \label must occur AFTER (or within) \caption.
	% For figures, \caption should occur after the \includegraphics.
	% Note that IEEEtran v1.7 and later has special internal code that
	% is designed to preserve the operation of \label within \caption
	% even when the captionsoff option is in effect. However, because
	% of issues like this, it may be the safest practice to put all your
	% \label just after \caption rather than within \caption{}.
	%
	% Reminder: the "draftcls" or "draftclsnofoot", not "draft", class
	% option should be used if it is desired that the figures are to be
	% displayed while in draft mode.
	%
	%\begin{figure}[!t]
	%\centering
	%\includegraphics[width=2.5in]{myfigure}
	% where an .eps filename suffix will be assumed under latex, 
	% and a .pdf suffix will be assumed for pdflatex; or what has been declared
	% via \DeclareGraphicsExtensions.
	%\caption{Simulation results for the network.}
	%\label{fig_sim}
	%\end{figure}
	
	% Note that the IEEE typically puts floats only at the top, even when this
	% results in a large percentage of a column being occupied by floats.
	
	
	% An example of a double column floating figure using two subfigures.
	% (The subfig.sty package must be loaded for this to work.)
	% The subfigure \label commands are set within each subfloat command,
	% and the \label for the overall figure must come after \caption.
	% \hfil is used as a separator to get equal spacing.
	% Watch out that the combined width of all the subfigures on a 
	% line do not exceed the text width or a line break will occur.
	%
	%\begin{figure*}[!t]
	%\centering
	%\subfloat[Case I]{\includegraphics[width=2.5in]{box}%
	%\label{fig_first_case}}
	%\hfil
	%\subfloat[Case II]{\includegraphics[width=2.5in]{box}%
	%\label{fig_second_case}}
	%\caption{Simulation results for the network.}
	%\label{fig_sim}
	%\end{figure*}
	%
	% Note that often IEEE papers with subfigures do not employ subfigure
	% captions (using the optional argument to \subfloat[]), but instead will
	% reference/describe all of them (a), (b), etc., within the main caption.
	% Be aware that for subfig.sty to generate the (a), (b), etc., subfigure
	% labels, the optional argument to \subfloat must be present. If a
	% subcaption is not desired, just leave its contents blank,
	% e.g., \subfloat[].
	
	
	% An example of a floating table. Note that, for IEEE style tables, the
	% \caption command should come BEFORE the table and, given that table
	% captions serve much like titles, are usually capitalized except for words
	% such as a, an, and, as, at, but, by, for, in, nor, of, on, or, the, to
	% and up, which are usually not capitalized unless they are the first or
	% last word of the caption. Table text will default to \footnotesize as
	% the IEEE normally uses this smaller font for tables.
	% The \label must come after \caption as always.
	%
	%\begin{table}[!t]
	%% increase table row spacing, adjust to taste
	%\renewcommand{\arraystretch}{1.3}
	% if using array.sty, it might be a good idea to tweak the value of
	% \extrarowheight as needed to properly center the text within the cells
	%\caption{An Example of a Table}
	%\label{table_example}
	%\centering
	%% Some packages, such as MDW tools, offer better commands for making tables
	%% than the plain LaTeX2e tabular which is used here.
	%\begin{tabular}{|c||c|}
	%\hline
	%One & Two\\
	%\hline
	%Three & Four\\
	%\hline
	%\end{tabular}
	%\end{table}
	
	
	% Note that the IEEE does not put floats in the very first column
	% - or typically anywhere on the first page for that matter. Also,
	% in-text middle ("here") positioning is typically not used, but it
	% is allowed and encouraged for Computer Society conferences (but
	% not Computer Society journals). Most IEEE journals/conferences use
	% top floats exclusively. 
	% Note that, LaTeX2e, unlike IEEE journals/conferences, places
	% footnotes above bottom floats. This can be corrected via the
	% \fnbelowfloat command of the stfloats package.
	
	
	% trigger a \newpage just before the given reference
	% number - used to balance the columns on the last page
	% adjust value as needed - may need to be readjusted if
	% the document is modified later
	%\IEEEtriggeratref{8}
	% The "triggered" command can be changed if desired:
	%\IEEEtriggercmd{\enlargethispage{-5in}}
	
	% references section
	
	% can use a bibliography generated by BibTeX as a .bbl file
	% BibTeX documentation can be easily obtained at:
	% http://mirror.ctan.org/biblio/bibtex/contrib/doc/
	% The IEEEtran BibTeX style support page is at:
	% http://www.michaelshell.org/tex/ieeetran/bibtex/
	%\bibliographystyle{IEEEtran}
	% argument is your BibTeX string definitions and bibliography database(s)
	%\bibliography{IEEEabrv,../bib/paper}
	%
	% <OR> manually copy in the resultant .bbl file
	% set second argument of \begin to the number of references
	% (used to reserve space for the reference number labels box)
	\bibliographystyle{IEEEtran}
	\bibliography{summarize}
	
	
\end{document}
