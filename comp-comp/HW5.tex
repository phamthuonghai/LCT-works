\documentclass{article}
\usepackage[utf8]{inputenc}
\usepackage{amssymb}
\usepackage{amsmath}
\usepackage{minted}

\DeclareMathOperator*{\argmax}{argmax}

\title{NTIN090 - HW5}
\author{Thuong-Hai Pham}
\date{January 2018}

\begin{document}

\maketitle

\section{Problem 8}

Prove that if there exists a polynomial algorithm deciding Hamiltonicity problem, then there
also exists an algorithm which finds a Hamiltonian cycle in a given graph in polynomial time (if a Hamiltonian
exists).

Assume that there exists a polynomial decider $IsHam(G)$ for Hamiltonicity problem given graph $G=(V,E)$, one can construct the program $FindHam(G)$ which finds a Hamiltonian cycle in $G$ as follow:

\begin{minted}[frame=single]{pascal}
program FindHam(G)
    res := []
    if IsHam(G) then
        return res

    u := arbitrary node in G
    while G has unchecked edges do
        for all edge e = (u, v) do
            Gt := G - e             // Remove e from graph G
            if IsHam(Gt) then
                G := Gt
            else
                add u to res
                mark e as checked
                next := v
        u := next
        
    return res
\end{minted}

The main procedure of $FindHam(G)$ is for a vertex $u$, delete all edge $e$ from $u$ that is not crucial to the Hamiltonian property (deleting $e$ from $G$ does not make $G$ not Hamiltonian). The deletion might delete edges from other Hamiltonian cycles, yet we only need one cycle.

The complexity of $FindHam$ is at most the multiplication of $|E|$ and the complexity of $IsHam$, which is polynomial. Hence, $FindHam$ is polynomial.


\section{Problem 9}

Consider the following weighted version of Sumset-Sum problem.

\textbf{Instance:} A set of $k$ items $A$, size $s(a)\in \mathbb{N}$ and value $v(a)\in \mathbb{N}$ associated with each item $a\in A$ and size
limit $l$.

\textbf{Feasible solution:} A set $A'\subseteq A$ satisfying $\sum_{a\in A'}s(a) \le l$.

\textbf{Objective:} Maximize the sum of values of items in $A'$, that is $\sum_{a\in A'}v(a)$.

Construct a pseudo-polynomial algorithm that finds the optimal solution. Construct a fully polynomial approximation scheme.

\subsection{Pseudo-polynomial algorithm to find the optimal solution}
Consider a Dynamic Programming algorithm that computes $F(i,v)$, minimum sum in size of items in $A''$, which $A''$ is a subset of $A[1..i]$ (the first i items of A) and $\displaystyle \sum_{a\in A''}v(a)=v$. Let $\displaystyle V=\max_{a\in A}(v(a))$, 

$$F(i,v)=min(F(i-1,v),F(i-1,v-v(A_i))+s(A_i)),\forall v\le kV, 1\le i\le k$$

For more technical details, $F(0,0)$ should be initialized as 0, others are $\infty$. The value of optimal solution is $\displaystyle \argmax_{v\le kV}(F(k,v)<l)$. While the set $O$ of optimal solution can be found by back-tracking.

It is obvious that the complexity of this algorithm is $O(k^2 V)$ which is polynomial to the length of inputs and the maximum value of inputs. Hence, the algorithm is pseudo-polynomial.

\subsection{Fully polynomial approximation scheme}

Given $$\epsilon > 0 \text{, let } \delta=\frac{\epsilon V}{k} \text{ and } v'(a)=\left \lfloor \frac{v(a)}{\delta} \right \rfloor, \forall a\in A$$.

Value of a set $$v(A)=\sum_{a\in A}v(a), v'(A)=\sum_{a\in A}v'(a)$$

We can use the Dynamic Programming algorithm in the previous section to find $O' \subseteq A$ that is in the set of feasible solution, and maximize the sum of new values (i.e. $v'(O')$). In addition, let O be the optimal solution found by the Dynamic Programming algorithm over the original values $v(a)$.

It is obvious that $v(a)-\delta v'(a)\le \delta, \forall a\in A$ (by scaling down and take the floor). Hence,

\begin{equation}
    v(O)-\delta v'(O)\le \delta k \\
    \Leftrightarrow \delta v'(O)\ge v(O)- \delta k
\end{equation}

Moreover, 
$$v(O')\ge \delta v'(O') \text{ because } v(a)\ge \delta v'(a),\forall a\in A$$

and

$$v'(O')\ge v'(O), (O' \text{ is the optimal solution over } v'(a))$$

Therefore,


\begin{equation}
\begin{aligned}
v(O')&\ge \delta v'(O) \\
    &\ge v(O) - \delta k \text{ , because of (1)} \\
    &\ge v(O) - \epsilon V
\end{aligned}
\end{equation}

WLOG, a set $\{a|a\in A,v(a)=V\}$ must be a feasible solution. Because if $s(a)>l$, any set that contains $a$ is also not a feasible solution, then we can remove these members of A before running the algorithm. Hence, $v(O)\ge V$.

$$(2) \Rightarrow v(O')\ge v(O)-\epsilon v(O) = (1-\epsilon)v(O)$$

which means the approximation ratio is $1-epsilon$, i.e. this is an approximation scheme for the problem.

On the other hand, the maximum value of inputs that the proposed algorithm has to work with is $\frac{k}{\epsilon}$. Therefore, the complexity is $O\left ( \frac{k^3}{\epsilon}\right )$. With that, we have proved the proposed algorithm is a fully polynomial time approximation scheme for the given problem.

\end{document}
