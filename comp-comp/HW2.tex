\documentclass{article}
\usepackage[utf8]{inputenc}
\usepackage{multirow}
\usepackage{amsmath}
\usepackage{amsfonts}
\usepackage{amssymb}
\usepackage{natbib}
\usepackage{graphicx}
\usepackage{amsthm}
\usepackage{minted}
\usemintedstyle{emacs}

\title{NTIN090 - HW2}
\author{Thuong-Hai Pham}
\date{November 2017}

\newtheorem{lemma}{Lemma}

\begin{document}

\maketitle

\section{Problem 6}
\textbf{\textit{Is the following language (partially) decidable?}}
\[B=\{(M, w); M \text{ terminates on input } w \text{ and the tape of } M \text{ is empty after the computation}\}\]

\textbf{Partial decidability}

WLOG, assume $M$ is 1-tape TM, consider the following multi-tape Turing Machine (TM) $RB(M,w)$:
\begin{enumerate}
    \item Copy $w$ to tape 3.
    \item Simulate $M(w)$  with tape 1 as the tape of $M$.
    \item For each transition with head movement, write the movement (L,R) to tape 2, move right on tape 2.
    \item When the simulation terminates, check if the tape is empty by:
        \begin{enumerate}
            \item Read tape 2 from right to left.
            \item For each `L' read, move right on tape 1, `R' read, move left.
            \item After tape 2, read tape 3 from right to left until blank, for each symbol on tape 3, move right on tape 1.
            \item On each move, check if the current cell is a non-blank symbol, then reject (terminate), else continue.
            \item After no non-blank symbol found, accept.
        \end{enumerate}
\end{enumerate}

Tape 2 and 3 help $RB$ to memorize which cell might be written (contains non-blank symbol) after the computation. If $M(w)$ terminates, the length of input word, number of L and R moves are finite, so there is always a decision for $(M,w)$. Hence, $RB$ accepts $(M,w)\in B$ and rejects or loops with $(M,w)\notin B$. Therefore, $B$ is partially decidable (recognizable) with $RB$ as its recognizer.

\bigskip
\textbf{Undecidability}

% Consider the complement of language $B$:
% \[\bar{B}=\{(M, w); M\text{ loops on } w \text{ or } M \text{ terminates on } w \text{ and the tape of } M \text{ is not empty after the computation}\}\]
% \[B'=\{(M, w); M \text{ terminates on } w \text{ and the tape of } M \text{ is not empty after the computation}\}\]

Consider the language $ACCEPT=\{(M,w); w\in L(M)\}$ and the TM below which computes function $f$:

\begin{listing}[ht]
	\begin{minted}[frame=single]{pascal}
program f(M):
    program N(w):
        if M(w) accepts:
            clean_tape // a finite process as in RB
        else
            write a non-blank symbol to tape
    return N
	\end{minted}
\end{listing}

$(M,w)\in ACCEPT \Leftrightarrow (N=f(M),w)\in B$ (i.e. $ACCEPT$ is m-reducible to $B$). Hence, if there exists a decider $DB$ for $B$, one can also use it to decide $ACCEPT$, which is a contradiction because $ACCEPT$ is undecidable. Consequently, $B$ is undecidable.

In conclusion, \textbf{$B$ is partially decidable but not decidable}.

\section{Problem 7}
\textbf{\textit{Are the following languages (partially) decidable where M is a (code of a) Turing machine and
$k \in \mathbb{N}$ and $|w|$ is the length of a word w.}}

\subsection{$B=\{w; |w| \le k\}$} \label{7.1}
It is trivial to construct a TM $M$ to be the decider of this language $B$ as follows:
\begin{itemize}
    \item Set of tape symbols $\Sigma$.
    \item Set of accepting states $F=\{q_0..q_k\}$.
    \item Set of states $Q=F \cup \{q_r\}$, $q_r$ is the rejecting state.
    \item Initial state $q_0$.
    \item Transition function: $\delta: Q \times \Sigma \to Q \times \Sigma \times \{R\} \cup \{\perp\}$.
    \begin{itemize}
        \item $\delta(q_i, s) = (q_{i+1}, s, R), \forall i=0..k-1,s\in\Sigma\setminus\{\lambda\}$.
        \item $\delta(q_i, \lambda) = \perp, \forall i=0..k$.
        \item $\delta(q_k, s) = (q_r, s, R), \forall s \in \Sigma\setminus\{\lambda\}$.
        \item $\delta(q_r, s) = \perp, \forall s \in \Sigma$
    \end{itemize}
\end{itemize}

The above TM $M$ always halts, and accepts (after at most k+1 transitions) or rejects (after k+2 transitions). Hence, language \textbf{$B$ is decidable}.



\subsection{$B=\{(M,k); |L(M)| \le k\}$} \label{7.2}

$|L(M)| \le k$ is a non-trivial property as:
\begin{itemize}
    \item M accepts nothing ($L(M)=\emptyset$) is in this class.
    \item M accepts everything ($L(M)=\Sigma^*$) is not in this class.
\end{itemize}

Hence, by Rice's Theorem, $B$ is undecidable.

On the other hand, consider the complement of $B$:

\[\bar{B}=\{(M,k); |L(M)| > k\}\],

Because the set of all finite words $\Sigma^*$ is countable, one can construct a recognizer $RB'$ for $\bar{B}$ as follow:

\begin{enumerate}
    \item Run the 1st step of $M$ on the 1st word.
    \item Run the 2nd step of $M$ on the 1st word, the 1st step of $M$ on the 2nd word.
    \item Run the the 3rd step of $M$ on the 1st word, the 2nd step of $M$ on 2nd word, the 1st step of $M$ on the 3rd word.
    \item ...
\end{enumerate}

If $(M,k)\in \bar{B}$ then after a finite number of steps, $M$ will accept word $k+1$  then $RB'$ accepts, otherwise, $RB'$ will loop forever. Hence, $\bar{B}$ is partially decidable. Assume that $B$ is also partially decidable,

$B,\bar{B}$ are partially decidable $\Leftrightarrow B$ is decidable,

which contradicts the undecidability proven by Rice's Theorem. Therefore, \textbf{$B$ is not partially decidable (and not decidable)}.


\subsection{$B=\{M; |L(M)| \ge 10\}$}

$|L(M)| \ge 10$ is a non-trivial property as:
\begin{itemize}
    \item M accepts only empty word ($L(M)=\{''\}$) is not in this class.
    \item M accepts everything ($L(M)=\Sigma^*$) is in this class.
\end{itemize}

Hence, by Rice's Theorem, $B$ is undecidable.

Using the same strategy as in \ref{7.2} (recognizer for $RB'$), one can construct a recognizer $RB$ for $B$. Therefore, \textbf{$B$ is partially decidable, and undecidable.}

\subsection{$B=\{M; L(M)=\{w; |w|=10\}\}$}
It is trivial to design a TM $DL$ that decide language $L=\{w; |w|=10\}$ by scanning through the whole input words as in \ref{7.1} with a modification that $q_{10}$ is the only accepting state, the others are rejecting states. Hence, $L$ is a decidable language. This property depicts a set of decidable languages, so $B$ is undecidable, a corollary of Rice's Theorem.

On the other hand, consider the complement of $B$:

\[\bar{B}=\{M; L(M)\text{ contains a word of length }\neq k\}\]

By the same strategy as in \ref{7.2}, a recognizer $RB'$ is able to accept a word of not length k after finite steps if $M\in \bar{B}$. Therefore, $\bar{B}$ is partially decidable. Assume that $B$ is also partially decidable leads to a contradiction as 

$B,\bar{B}$ are both partially decidable $\Leftrightarrow B$ is decidable.

Hence, \textbf{$B$ is not partially decidable (and not decidable)}.

% Consider the language $EMPTY=\{M; L(M)=\emptyset\}$.


\subsection{$B=\{(M,k); L(M) \textnormal{ contains a word of length } k\}$}

``$L(M)$ contains a word of length $k$" is a non-trivial property as:
\begin{itemize}
    \item M accepts nothing ($L(M)=\emptyset$) is not in this class.
    \item M accepts k-length words ($L(M)=\Sigma^k$) is in this class.
\end{itemize}

Hence, by Rice's Theorem, $B$ is undecidable.

Using the same strategy as in \ref{7.2}, one can construct a recognizer $RB$ for this language $B$. If $(M,k)\in B$, then after a finite number of steps, $RB$ will halt and accept $(M,k)$. Therefore, \textbf{$B$ is partially decidable, and undecidable.}


\end{document}
